\documentclass[a4paper, 14pt]{extarticle}
\usepackage[russian]{babel}
\usepackage[T1]{fontenc}
\usepackage{fontspec}
\usepackage{indentfirst}
\usepackage{enumitem}
\usepackage{graphicx}
\usepackage{scrextend}
\usepackage{longtable}
\usepackage[
  left=20mm,
  right=10mm,
  top=20mm,
  bottom=20mm
]{geometry}
\usepackage{parskip}
\usepackage{titlesec}
\usepackage{xurl}
\usepackage{hyperref}
\usepackage{float}
\usepackage[
  figurename=Рисунок,
  labelsep=endash,
]{caption}
\usepackage[outputdir=build, newfloat]{minted}
\usepackage{multirow}
\usepackage{array}

\hypersetup{
  colorlinks=true,
  linkcolor=black,
  filecolor=blue,
  urlcolor=blue,
}

\renewcommand*{\labelitemi}{---}
\setmainfont{Times New Roman}
\setmonofont{JetBrains Mono}[
  SizeFeatures={Size=11},
]

\newenvironment{code}{\captionsetup{type=listing}}{}
\SetupFloatingEnvironment{listing}{name=Листинг}

\setminted{
  fontsize=\footnotesize,
  framesep=0mm,
}

\captionsetup{width=\textwidth,justification=centering}
\captionsetup[table]{singlelinecheck=off,justification=justified}

\newcolumntype{L}[1]{>{\raggedright\let\newline\\\arraybackslash\hspace{0pt}}m{#1}}
\newcolumntype{C}[1]{>{\centering\let\newline\\\arraybackslash\hspace{0pt}}m{#1}}
\newcolumntype{R}[1]{>{\raggedleft\let\newline\\\arraybackslash\hspace{0pt}}m{#1}}

\setlength{\parskip}{6pt}

\setlength{\parindent}{1cm}
\setlist[itemize]{itemsep=0em,topsep=0em,parsep=0em,partopsep=0em,leftmargin=2.0cm}
\setlist[enumerate]{itemsep=0em,topsep=0em,parsep=0em,partopsep=0em,leftmargin=2.0cm}

\renewcommand{\thesection}{\arabic{section}.}
\renewcommand{\thesubsection}{\thesection\arabic{subsection}.}
\renewcommand{\thesubsubsection}{\thesubsection\arabic{subsubsection}.}

\titleformat{\section}{\normalfont\bfseries}{\thesection}{0.5em}{}
\titleformat{\subsection}{\normalfont\bfseries}{\thesubsection}{0.5em}{}

\titleformat*{\section}{\normalfont\bfseries}
\titleformat*{\subsection}{\normalfont\bfseries}

\linespread{1.5}
\renewcommand{\baselinestretch}{1.5}

\renewcommand{\theenumii}{(\asbuk{enumii})}
\renewcommand{\labelenumii}{\asbuk{enumii})}
\AddEnumerateCounter{\asbuk}{\@asbuk}{\cyrm}

\begin{document}

\begin{titlepage}
  \vspace{0pt plus2fill}
  \noindent

  \vspace{0pt plus6fill}
  \begin{center}
    Санкт-Петербургский национальный исследовательский университет
    информационных технологий, механики и оптики

    \vspace{0pt plus2fill}

    Факультет инфокоммуникационных технологий

    Направление подготовки 11.03.02

    \vspace{0pt plus4fill}

    Практическая работа по теме:

    <<Единая государственная система абитуриента>>

  \end{center}

  \vspace{0pt plus7fill}
  \begin{flushright}
    Выполнил: \\
    Швалов Даниил Андреевич

    Группа: К33211

    Проверил: \\
    Осипов Никита Алексеевич
  \end{flushright}

  \vspace{0pt plus2fill}
  \begin{center}
    Санкт-Петербург

    2023
  \end{center}
\end{titlepage}

\setcounter{page}{2}

\section{Введение}

В данной практической работе необходимо спроектировать единую государственную
систему абитуриента (ЕГСА), которая позволила бы упростить процесс поступления.
Данная система выступает в роли агрегатора, предоставляющего актуальные данные,
интересующие абитуриента из многих ВУЗов, в рамках единой платформы. На портале
может быть отражена информация о позиции в рейтинге поступающих, информация об
учебном заведении и направлениях обучения в нем, отзывы выпускников и др.

Также системой предусматривается возможность дистанционной подачи заявления
абитуриентом. Единая государственная система абитуриента способна решить
проблемы не только будущих студентов, но и руководства учебных заведений, т.к.
часть документальной работы автоматизируется сервисом. Так, например, система
предусматривает возможность получения отчетностей по приемным кампаниям.

Данный программный продукт позволил бы устранить проблему отсутствия
стандартизации, поскольку информация о всех учебных заведениях была бы
сосредоточена в рамках единого портала. Такое решение позволяет сделать более
весомым фактором именно качество информации, а не способы ее представления в
пользовательском интерфейсе конкретного технологического решения ВУЗа.

\section{Анализ потребностей и возможностей создания программных систем}

\subsection*{Введение}

\textbf{Цель этапа}: Для выбранного варианта инфокоммуникационной системы
определить набор требований и спецификаций на создание системы, на основании
которых будет организована реализация проекта.

\subsection*{Задание 1. Реализация начальной фазы проекта}

\textbf{Цель}: ознакомиться с предложенным вариантом описания предметной
области. Проанализировать предметную область, уточнив и дополнив ее,
руководствуясь собственным опытом, консультациями и другими источниками.
Разработать описание замысла проекта и начального варианта модели прецедентов.

\textbf{Для} абитуриентов, \textbf{которым} необходимо определиться с высшим
учебным заведением, \textbf{данная} единая государственная система абитуриента
\textbf{является} информационной системой, \textbf{которая} обеспечит единую
точку доступа к подробной информации о каждом ВУЗе страны. Система будет
предоставлять информацию о том, что требуется для поступления на какое-нибудь
направление, какие условия предоставляет ВУЗ своим абитуриентам. Также система
способна решить проблемы не только будущих студентов, но и руководства учебных
заведений, т.к. часть документальной работы будет автоматизирована системой
(дистанционная подача заявления абитуриентом, получение отчетностей по приемным
кампаниям). \textbf{В отличие от} существующих систем, которые содержат мало
информации о ВУЗах и не позволяют напрямую взаимодействовать с ВУЗами,
\textbf{наш продукт} предоставит удобную систему, которая поможет абитуриентам
найти ВУЗ по интересам и знаниям, а также упростит процесс приемных кампаний
руководствам учебных заведений.

\subsection*{
  Задание 2. Создание прецедентов на уровне элементарных бизнес-процессов (EBP)
}

\textbf{Цель}: Исследовать задачи пользователей и сформулировать основные и
вспомогательные прецеденты на уровне элементарных бизнес-процессов (EBP).

Как абитуриент:
\begin{itemize}
  \item я хочу видеть требования, которые выставляет ВУЗ, для поступления, чтобы
  понимать, что я и мои знания соответствуют требованиям ВУЗа;
  \item я хочу знать, какие индивидуальные достижения учитываются при
  поступлении в данный ВУЗ, чтобы понимать, сколько итоговых баллов у меня будет
  при поступлении;
  \item я хочу знать, какое количество мест предоставляет ВУЗ для данного
  направления, чтобы понимать, смогу ли я попасть на это направление;
  \item я хочу видеть интерактивный список студентов, которые подали документы
  на это направление в данный ВУЗ, чтобы понимать, хватит ли мне баллов для
  прохождения на данное направление;
  \item я хочу видеть олимпиады, которые принимает ВУЗ, чтобы понимать, могу ли
  я поступить без вступительных испытаний;
  \item я хочу знать, требуется ли сдавать дополнительные вступительные
  испытания, чтобы понимать, нужно ли готовиться к дополнительным экзаменам
  перед поступлением;
  \item я хочу видеть, предоставляемые ВУЗом условия, такие как общежитие,
  военная кафедра и т. п., чтобы понимать, что ВУЗ удовлетворяет моим
  пожеланиям;
  \item я хочу видеть отзывы от реальных студентов о процессах обучения в данном
  ВУЗе, чтобы иметь понимание о реальном качестве обучения в этом ВУЗе;
  \item я хочу иметь возможность удаленно подавать заявление на зачисление в
  данный ВУЗ, чтобы не ехать в ВУЗ лично, а делать все через Интернет.
\end{itemize}

Как руководство учебного заведения:
\begin{itemize}
  \item я хочу видеть количество студентов, которое собирается поступить на
  данное направление, чтобы понимать примерное количество будущих студентов;
  \item я хочу предоставлять наиболее актуальную информацию для каждого
  направления и о ВУЗе в целом, чтобы абитуриенты имели более правдоподобную
  картину о ВУЗе;
  \item я хочу иметь возможность принимать заявления на поступление от
  абитуриентов в удаленном формате, чтобы упростить процесс приемной кампании;
  \item я хочу иметь возможность получать отчетность по приемным кампаниям,
  чтобы упростить процесс обработки и составления документов и отчетов по
  приемным кампаниям.
\end{itemize}

\subsection*{
  Задание 3. Построение и исследование моделей сценария использования (Use Case)
}

\textbf{Цель}: для выбранного варианта инфокоммуникационной системы реализовать
развернутое описание основного прецедента, на основании которых будет
организована реализация проекта. Разработать развернутое описание основного
прецедента.

\subsubsection*{Прецедент. Подача документов на поступление в ВУЗ}

{
  \setlength{\parindent}{0pt}

  \textbf{Основной исполнитель}. Абитуриент.

  \textbf{Заинтересованные лица и их требования}
  \begin{itemize}
    \item Абитуриент. Хочет быстро и просто внести в систему свои документы,
    выбрать ВУЗ по его знаниям и интересам, подать заявление и увидеть свои
    шансы на поступления.
    \item Руководство учебного заведения. Хочет видеть список всех поступающих
    абитуриентов, иметь систему, автоматизирующую задачи приемной кампании.
  \end{itemize}

  \textbf{Предусловия}. Абитуриент идентифицирован и аутентифицирован в системе.

  \textbf{Результаты (постусловия)}. Данные о подаче заявления сохранены.
  Документы, предоставленные абитуриентом, также сохранены, защищены и
  направлены в ВУЗ, в который абитуриент подал документы. Информация о
  количестве поданных заявлений на данное направление в данном ВУЗе обновлена.
  Рейтинг абитуриентов пересчитан.

  \textbf{Основной успешный сценарий (или основной процесс)}:
  \begin{enumerate}
    \item абитуриент заходит на сайт и авторизуется в системе;
    \item абитуриент загружает все необходимые документы о себе: аттестат,
    информация личных достижений и т. п.;
    \item система сохраняет информацию о абитуриенте;
    \item абитуриент переходит на страницу со списком ВУЗов, фильтрует
    ВУЗы по интересам, направлениям, потребностям, знаниям и отзывам студентов;
    \item абитуриент выбирает один или несколько ВУЗов (до 3 шт.), а также
    направления (также до 3 шт.), куда он хотел бы поступить;
    \item абитуриент подписывает заявление на поступление в данные ВУЗы;
    \item система обрабатывает заявления от абитуриента, отправляет информацию
    руководствам учебных заведений;
    \item руководства учебных заведений подтверждают прием документов и
    заявления от абитуриента;
    \item система добавляет абитуриента в очередь на поступление в ВУЗ.
  \end{enumerate}

  \textbf{Расширения (или альтернативные потоки)}:
  \begin{enumerate}
    \item[2а.] При неправильной загрузке документов абитуриента (неправильно сделана
      фотография, не загружены подтверждения личных достижений и т. п.) или при
      дополнительных требованиях ВУЗа.
      \begin{enumerate}
        \item абитуриент догружает все недостающие или неправильно сделанные
        документы и информацию о себе;
        \item руководство учебного заведения перепроверяет новую загруженную
        информацию.
      \end{enumerate}
    \item[3а.] Ошибка при сохранении информации об абитуриенте.
      \begin{enumerate}
        \item система уведомляет об ошибке пользователя и переходит в предыдущее
        состояние;
        \item пользователь пытается снова загрузить документы.
      \end{enumerate}
    \item[6-8а.] Абитуриент хочет отозвать заявление из ВУЗа.
      \begin{enumerate}
        \item абитуриент выбирает заявление, которое нужно отменить;
        \item система получает информацию об отмене заявления, направляет
        уведомление руководству учебного заведения;
        \item руководство учебного заведения фиксирует отмену заявления;
        \item система обновляет список абитуриентов, подавших заявление в ВУЗ.
      \end{enumerate}
  \end{enumerate}

  \textbf{Специальные требования}:
  \begin{itemize}
    \item все документы и информация о абитуриентах должны быть защищены и
    храниться в зашифрованном виде;
    \item система должна выдерживать высокие нагрузки: использование системы
    большим количеством пользователей единовременно не должно препятствовать
    работе системы;
    \item при сбое системы никакие данные не должны быть повреждены, все
    операции, выполняемые во время сбоя, должны быть отменены без повреждения
    данных.
  \end{itemize}

  \textbf{Список технологий и типов данных}:
  \begin{itemize}
    \item доступ систему должен быть возможен со всех основных типов устройств и
    операционных систем (iOS, Android, Windows, Linux, macOS);
    \item база данных, используемая для хранения данных об абитуриентах, должна
    соответствовать требованиям ACID, а также обладать механизмом транзакций.
  \end{itemize}

  \textbf{Частота использования}:
  \begin{itemize}
    \item для абитуриентов: от 1 до 5 раз за все время;
    \item для руководств учебных заведений: ежегодно во время работы приемной
    кампании.
  \end{itemize}

  \textbf{Открытые вопросы}:
  \begin{itemize}
    \item изучить нормативные документы по организации процессов приемных
    кампаний разных ВУЗов;
    \item исследовать максимально возможную нагрузку в пик подачи заявлений во
    время приемной кампании;
    \item выяснить, какое количество данных будет хранить система.
  \end{itemize}
}

\section{Этапы разработки проекта: стратегия и анализ}

\subsection*{Введение}

\textbf{Цель этапа}: для выбранного варианта инфокоммуникационной системы
реализовать программные документы на создание системы, на основании которых
будет организована реализация проекта.

\subsection*{Задание 4. Создание документа-концепции}

\textbf{Цель}: разработать дополнительную спецификацию, словарь терминов и
документ-концепцию.

\subsubsection*{Введение}

% TODO:

\subsubsection*{Позиционирование}

\paragraph*{Экономические предпосылки}

% TODO:

\paragraph*{Формулировка проблемы}

% TODO:

\paragraph*{Место системы}

% TODO:

\paragraph*{Заинтересованные лица}

% TODO:

\paragraph*{Основные задачи высокого уровня и проблемы заинтересованных лиц}

% TODO:

\begin{longtable}{|L{0.225\textwidth}|L{0.225\textwidth}|L{0.225\textwidth}|L{0.225\textwidth}|}
  \hline
  \textbf{Цель высокого уровня} & \textbf{Приоритет} & \textbf{Проблемы и замечания} & \textbf{Текущие решения} \\
  \hline
  % TODO:
\end{longtable}

\paragraph*{Задачи уровня пользователя}

% TODO:

\subsubsection*{Обзор}

\paragraph*{Перспективы продукта}

% TODO:

\paragraph*{Преимущества системы}

% TODO:

\begin{longtable}{|L{0.47\textwidth}|L{0.47\textwidth}|}
  \hline
  \textbf{Свойство} & \textbf{Преимущества для заинтересованных лиц} \\
  \hline
  % TODO:
\end{longtable}

\paragraph*{Основные свойства системы}

% TODO:

\paragraph*{Другие требования и ограничения}

% TODO:

\subsection*{
  Задание 5. Разработка спецификации требований к программному обеспечению
  (Modern Software Requirements Specification)
}

\textbf{Цель}: изучить структуру документа SRS, разработать текст спецификации
согласно шаблона.

% TODO:

\subsection*{Задание 6. Разработка технического задания}

\textbf{Цель}: разработать техническое задание на создание автоматизированной
системы согласно ГОСТ 34.602-89.

% TODO:

\section{Стадия развития}

\subsection*{Задание 7. Реализация диаграммы взаимодействия}

\textbf{Цель}: освоить методику проектирования реализации прецедентов, изучить
шаблоны GRASP для распределения обязанностей между классами.

% TODO:

\subsection*{
  Задание 8. Построение и исследование программной системы на основе шаблонов
  проектирования
}

\textbf{Цель}: изучить шаблоны проектирования GoF и закрепить основы разработки
систем на их основе, освоить IDE MS Visual Studio в части разработки диаграммы
классов.

% TODO:

\end{document}