\documentclass[a4paper, 14pt]{extarticle}
\usepackage[russian]{babel}
\usepackage[T1]{fontenc}
\usepackage{fontspec}
\usepackage{indentfirst}
\usepackage{enumitem}
\usepackage{graphicx}
\usepackage{scrextend}
\usepackage{longtable}
\usepackage[
  left=20mm,
  right=10mm,
  top=20mm,
  bottom=20mm
]{geometry}
\usepackage{parskip}
\usepackage{titlesec}
\usepackage{xurl}
\usepackage{hyperref}
\usepackage{float}
\usepackage[
  figurename=Рисунок,
  labelsep=endash,
]{caption}
\usepackage[outputdir=build, newfloat]{minted}
\usepackage{multirow}
\usepackage{array}

\hypersetup{
  colorlinks=true,
  linkcolor=black,
  filecolor=blue,
  urlcolor=blue,
}

\renewcommand*{\labelitemi}{---}
\setmainfont{Times New Roman}
\setmonofont{JetBrains Mono}[
  SizeFeatures={Size=11},
]

\newenvironment{code}{\captionsetup{type=listing}}{}
\SetupFloatingEnvironment{listing}{name=Листинг}

\setminted{
  fontsize=\footnotesize,
  framesep=0mm,
}

\captionsetup{width=\textwidth,justification=centering}
\captionsetup[table]{singlelinecheck=off,justification=justified}

\newcolumntype{L}[1]{>{\raggedright\let\newline\\\arraybackslash\hspace{0pt}}m{#1}}
\newcolumntype{C}[1]{>{\centering\let\newline\\\arraybackslash\hspace{0pt}}m{#1}}
\newcolumntype{R}[1]{>{\raggedleft\let\newline\\\arraybackslash\hspace{0pt}}m{#1}}

\setlength{\parskip}{6pt}

\setlength{\parindent}{1cm}
\setlist[itemize]{itemsep=0em,topsep=0em,parsep=0em,partopsep=0em,leftmargin=2.0cm}
\setlist[enumerate]{itemsep=0em,topsep=0em,parsep=0em,partopsep=0em,leftmargin=2.0cm}

\renewcommand{\thesection}{\arabic{section}.}
\renewcommand{\thesubsection}{\thesection\arabic{subsection}.}
\renewcommand{\thesubsubsection}{\thesubsection\arabic{subsubsection}.}

\titleformat{\section}{\normalfont\bfseries}{\thesection}{0.5em}{}
\titleformat{\subsection}{\normalfont\bfseries}{\thesubsection}{0.5em}{}

\titleformat*{\section}{\normalfont\bfseries}
\titleformat*{\subsection}{\normalfont\bfseries}

\linespread{1.5}
\renewcommand{\baselinestretch}{1.5}

\renewcommand{\theenumii}{(\asbuk{enumii})}
\renewcommand{\labelenumii}{\asbuk{enumii})}
\AddEnumerateCounter{\asbuk}{\@asbuk}{\cyrm}

\begin{document}

\begin{titlepage}
  \vspace{0pt plus2fill}
  \noindent

  \vspace{0pt plus6fill}
  \begin{center}
    Санкт-Петербургский национальный исследовательский университет
    информационных технологий, механики и оптики

    \vspace{0pt plus2fill}

    Практическая работа по теме:

    <<Единая государственная система абитуриента>>

    \vspace{0pt plus1fill}

    Задание №2

    <<Создание прецедентов на уровне элементарных бизнес-процессов (EBP)>>

  \end{center}

  \vspace{0pt plus7fill}
  \begin{flushright}
    Выполнил: \\
    Швалов Даниил Андреевич

    Группа: К33211

    Проверил: \\
    Осипов Никита Алексеевич
  \end{flushright}

  \vspace{0pt plus2fill}
  \begin{center}
    Санкт-Петербург

    2023
  \end{center}
\end{titlepage}

\setcounter{page}{2}

\section{Введение}

\textbf{Цель}: исследовать задачи пользователей и сформулировать основные и
вспомогательные прецеденты на уровне элементарных бизнес-процессов (EBP).

\section{Ход работы}

В ходе исследования задач пользователей были сформулированы следующие прецеденты.

Как абитуриент:
\begin{itemize}
  \item я хочу видеть требования, которые выставляет ВУЗ, для поступления, чтобы
  понимать, что я и мои знания соответствуют требованиям ВУЗа;
  \item я хочу знать, какие индивидуальные достижения учитываются при
  поступлении в данный ВУЗ, чтобы понимать, сколько итоговых баллов у меня будет
  при поступлении;
  \item я хочу знать, какое количество мест предоставляет ВУЗ для данного
  направления, чтобы понимать, смогу ли я попасть на это направление;
  \item я хочу видеть интерактивный список студентов, которые подали документы
  на это направление в данный ВУЗ, чтобы понимать, хватит ли мне баллов для
  прохождения на данное направление;
  \item я хочу видеть олимпиады, которые принимает ВУЗ, чтобы понимать, могу ли
  я поступить без вступительных испытаний;
  \item я хочу знать, требуется ли сдавать дополнительные вступительные
  испытания, чтобы понимать, нужно ли готовиться к дополнительным экзаменам
  перед поступлением;
  \item я хочу видеть, предоставляемые ВУЗом условия, такие как общежитие,
  военная кафедра и т. п., чтобы понимать, что ВУЗ удовлетворяет моим
  пожеланиям;
  \item я хочу видеть отзывы от реальных студентов о процессах обучения в данном
  ВУЗе, чтобы иметь понимание о реальном качестве обучения в этом ВУЗе;
  \item я хочу иметь возможность удаленно подавать заявление на зачисление в
  данный ВУЗ, чтобы не ехать в ВУЗ лично, а делать все через Интернет.
\end{itemize}

Как руководство учебного заведения:
\begin{itemize}
  \item я хочу видеть количество студентов, которое собирается поступить на
  данное направление, чтобы понимать примерное количество будущих студентов;
  \item я хочу предоставлять наиболее актуальную информацию для каждого
  направления и о ВУЗе в целом, чтобы абитуриенты имели более правдоподобную
  картину о ВУЗе;
  \item я хочу иметь возможность принимать заявления на поступление от
  абитуриентов в удаленном формате, чтобы упростить процесс приемной кампании;
  \item я хочу иметь возможность получать отчетность по приемным кампаниям,
  чтобы упростить процесс обработки и составления документов и отчетов по
  приемным кампаниям.
\end{itemize}

\section{Заключение}

В ходе выполнения данного практического задания я исследовал задачи
пользователей и сформулировал основные и вспомогательные прецеденты на уровне
элементарных бизнес-процессов (EBP).

\end{document}