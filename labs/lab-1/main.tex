\documentclass[a4paper, 14pt]{extarticle}
\usepackage[russian]{babel}
\usepackage[T1]{fontenc}
\usepackage{fontspec}
\usepackage{indentfirst}
\usepackage{enumitem}
\usepackage{graphicx}
\usepackage{scrextend}
\usepackage{longtable}
\usepackage[
  left=20mm,
  right=10mm,
  top=20mm,
  bottom=20mm
]{geometry}
\usepackage{parskip}
\usepackage{titlesec}
\usepackage{xurl}
\usepackage{hyperref}
\usepackage{float}
\usepackage[
  figurename=Рисунок,
  labelsep=endash,
]{caption}
\usepackage[outputdir=build, newfloat]{minted}
\usepackage{multirow}
\usepackage{array}

\hypersetup{
  colorlinks=true,
  linkcolor=black,
  filecolor=blue,
  urlcolor=blue,
}

\renewcommand*{\labelitemi}{---}
\setmainfont{Times New Roman}
\setmonofont{JetBrains Mono}[
  SizeFeatures={Size=11},
]

\newenvironment{code}{\captionsetup{type=listing}}{}
\SetupFloatingEnvironment{listing}{name=Листинг}

\setminted{
  fontsize=\footnotesize,
  framesep=0mm,
}

\captionsetup{width=\textwidth,justification=centering}
\captionsetup[table]{singlelinecheck=off,justification=justified}

\newcolumntype{L}[1]{>{\raggedright\let\newline\\\arraybackslash\hspace{0pt}}m{#1}}
\newcolumntype{C}[1]{>{\centering\let\newline\\\arraybackslash\hspace{0pt}}m{#1}}
\newcolumntype{R}[1]{>{\raggedleft\let\newline\\\arraybackslash\hspace{0pt}}m{#1}}

\setlength{\parskip}{6pt}

\setlength{\parindent}{1cm}
\setlist[itemize]{itemsep=0em,topsep=0em,parsep=0em,partopsep=0em,leftmargin=2.0cm}
\setlist[enumerate]{itemsep=0em,topsep=0em,parsep=0em,partopsep=0em,leftmargin=2.0cm}

\renewcommand{\thesection}{\arabic{section}.}
\renewcommand{\thesubsection}{\thesection\arabic{subsection}.}
\renewcommand{\thesubsubsection}{\thesubsection\arabic{subsubsection}.}

\titleformat{\section}{\normalfont\bfseries}{\thesection}{0.5em}{}
\titleformat{\subsection}{\normalfont\bfseries}{\thesubsection}{0.5em}{}

\titleformat*{\section}{\normalfont\bfseries}
\titleformat*{\subsection}{\normalfont\bfseries}

\linespread{1.5}
\renewcommand{\baselinestretch}{1.5}

\renewcommand{\theenumii}{(\asbuk{enumii})}
\renewcommand{\labelenumii}{\asbuk{enumii})}
\AddEnumerateCounter{\asbuk}{\@asbuk}{\cyrm}

\begin{document}

\begin{titlepage}
  \vspace{0pt plus2fill}
  \noindent

  \vspace{0pt plus6fill}
  \begin{center}
    Санкт-Петербургский национальный исследовательский университет
    информационных технологий, механики и оптики

    \vspace{0pt plus2fill}

    Практическая работа по теме:

    <<Единая государственная система абитуриента>>

    \vspace{0pt plus1fill}

    Задание №1

    <<Реализация начальной фазы проекта>>

  \end{center}

  \vspace{0pt plus7fill}
  \begin{flushright}
    Выполнил: \\
    Швалов Даниил Андреевич

    Группа: К33211

    Проверил: \\
    Осипов Никита Алексеевич
  \end{flushright}

  \vspace{0pt plus2fill}
  \begin{center}
    Санкт-Петербург

    2023
  \end{center}
\end{titlepage}

\setcounter{page}{2}

\section{Введение}

\textbf{Цель}: ознакомиться с предложенным вариантом описания предметной
области. Проанализировать предметную область, уточнив и дополнив ее,
руководствуясь собственным опытом, консультациями и другими источниками.
Разработать описание замысла проекта и начального варианта модели прецедентов.

\section{Ход работы}

В данной практической работе необходимо спроектировать единую государственную
систему абитуриента (ЕГСА), которая позволила бы упростить процесс поступления.
Данная система выступает в роли агрегатора, предоставляющего актуальные данные,
интересующие абитуриента из многих ВУЗов, в рамках единой платформы. На портале
может быть отражена информация о позиции в рейтинге поступающих, информация об
учебном заведении и направлениях обучения в нем, отзывы выпускников и др.

Также системой предусматривается возможность дистанционной подачи заявления
абитуриентом. Единая государственная система абитуриента способна решить
проблемы не только будущих студентов, но и руководства учебных заведений, т.к.
часть документальной работы автоматизируется сервисом. Так, например, система
предусматривает возможность получения отчетностей по приемным кампаниям.

Данный программный продукт позволил бы устранить проблему отсутствия
стандартизации, поскольку информация о всех учебных заведениях была бы
сосредоточена в рамках единого портала. Такое решение позволяет сделать более
весомым фактором именно качество информации, а не способы ее представления в
пользовательском интерфейсе конкретного технологического решения ВУЗа.

В ходе анализа предметной области было сформировано следующее описание замысла
проекта и начального варианта модели прецедентов.

\textbf{Для} абитуриентов, \textbf{которым} необходимо определиться с высшим
учебным заведением, \textbf{данная} единая государственная система абитуриента
\textbf{является} информационной системой, \textbf{которая} обеспечит единую
точку доступа к подробной информации о каждом ВУЗе страны. Система будет
предоставлять информацию о том, что требуется для поступления на какое-нибудь
направление, какие условия предоставляет ВУЗ своим абитуриентам. Также система
способна решить проблемы не только будущих студентов, но и руководства учебных
заведений, т.к. часть документальной работы будет автоматизирована системой
(дистанционная подача заявления абитуриентом, получение отчетностей по приемным
кампаниям). \textbf{В отличие от} существующих систем, которые содержат мало
информации о ВУЗах и не позволяют напрямую взаимодействовать с ВУЗами,
\textbf{наш продукт} предоставит удобную систему, которая поможет абитуриентам
найти ВУЗ по интересам и знаниям, а также упростит процесс приемных кампаний
руководствам учебных заведений.

\section{Заключение}

В ходе выполнения данного практического задания я ознакомился с предложенным
вариантом описания предметной области, проанализировать предметную область,
уточнив и дополнив ее, руководствуясь собственным опытом, консультациями и
другими источниками, а также разработал описание замысла проекта и начального
варианта модели прецедентов.

\end{document}