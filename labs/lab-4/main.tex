\documentclass[a4paper, 14pt]{extarticle}
\usepackage[russian]{babel}
\usepackage[T1]{fontenc}
\usepackage{fontspec}
\usepackage{indentfirst}
\usepackage{enumitem}
\usepackage{graphicx}
\usepackage{scrextend}
\usepackage{longtable}
\usepackage[
  left=20mm,
  right=10mm,
  top=20mm,
  bottom=20mm
]{geometry}
\usepackage{parskip}
\usepackage{titlesec}
\usepackage{xurl}
\usepackage{hyperref}
\usepackage{float}
\usepackage[
  figurename=Рисунок,
  labelsep=endash,
]{caption}
\usepackage[outputdir=build, newfloat]{minted}
\usepackage{multirow}
\usepackage{array}

\hypersetup{
  colorlinks=true,
  linkcolor=black,
  filecolor=blue,
  urlcolor=blue,
}

\renewcommand*{\labelitemi}{---}
\setmainfont{Times New Roman}
\setmonofont{JetBrains Mono}[
  SizeFeatures={Size=11},
]

\newenvironment{code}{\captionsetup{type=listing}}{}
\SetupFloatingEnvironment{listing}{name=Листинг}

\setminted{
  fontsize=\footnotesize,
  framesep=0mm,
}

\captionsetup{width=\textwidth,justification=centering}
\captionsetup[table]{singlelinecheck=off,justification=justified}

\newcolumntype{L}[1]{>{\raggedright\let\newline\\\arraybackslash\hspace{0pt}}m{#1}}
\newcolumntype{C}[1]{>{\centering\let\newline\\\arraybackslash\hspace{0pt}}m{#1}}
\newcolumntype{R}[1]{>{\raggedleft\let\newline\\\arraybackslash\hspace{0pt}}m{#1}}

\setlength{\parskip}{6pt}

\setlength{\parindent}{1cm}
\setlist[itemize]{itemsep=0em,topsep=0em,parsep=0em,partopsep=0em,leftmargin=2.0cm}
\setlist[enumerate]{itemsep=0em,topsep=0em,parsep=0em,partopsep=0em,leftmargin=2.0cm}

\renewcommand{\thesection}{\arabic{section}.}
\renewcommand{\thesubsection}{\thesection\arabic{subsection}.}
\renewcommand{\thesubsubsection}{\thesubsection\arabic{subsubsection}.}

\titleformat{\section}{\normalfont\bfseries}{\thesection}{0.5em}{}
\titleformat{\subsection}{\normalfont\bfseries}{\thesubsection}{0.5em}{}

\titleformat*{\section}{\normalfont\bfseries}
\titleformat*{\subsection}{\normalfont\bfseries}

\linespread{1.5}
\renewcommand{\baselinestretch}{1.5}

\renewcommand{\theenumii}{(\asbuk{enumii})}
\renewcommand{\labelenumii}{\asbuk{enumii})}
\AddEnumerateCounter{\asbuk}{\@asbuk}{\cyrm}

\begin{document}

\begin{titlepage}
  \vspace{0pt plus2fill}
  \noindent

  \vspace{0pt plus6fill}
  \begin{center}
    Санкт-Петербургский национальный исследовательский университет
    информационных технологий, механики и оптики

    \vspace{0pt plus2fill}

    Практическая работа по теме:

    <<Единая государственная система абитуриента>>

    \vspace{0pt plus1fill}

    Задание №4

    <<Создание документа-концепции>>

  \end{center}

  \vspace{0pt plus7fill}
  \begin{flushright}
    Выполнил: \\
    Швалов Даниил Андреевич

    Группа: К33211

    Проверил: \\
    Осипов Никита Алексеевич
  \end{flushright}

  \vspace{0pt plus2fill}
  \begin{center}
    Санкт-Петербург

    2023
  \end{center}
\end{titlepage}

\setcounter{page}{2}

\section{Введение}

\textbf{Цель}: разработать дополнительную спецификацию, документ-концепцию и
словарь терминов.

\newpage

\section{Дополнительная спецификация}

\subsection*{Дата внесения изменений}

\begin{table}[H]
  \begin{tabular}{|L{0.17\textwidth}|L{0.19\textwidth}|L{0.4\textwidth}|L{0.15\textwidth}|}
    \hline
    \textbf{Версия}                                                  &
    \textbf{Дата}                                                    &
    \textbf{Описание}                                                &
    \textbf{Автор}                                                     \\
    \hline
    Черновой начальный вариант                                       &
    31 декабря, 2023                                                 &
    Первый черновой вариант. Будет уточнен на первой стадии развития &
    Швалов Даниил                                                      \\
    \hline
  \end{tabular}
\end{table}

\subsection*{Введение}

В этом документе описаны все требования к Единой государственной системе
абитуриента, не вошедшие в описания прецедентов.

\subsection*{Функциональность}

\subsubsection*{Регистрация событий и обработка ошибок}

Все события пользователей и систем журналируются и сохраняются на нескольких
постоянных носителях для обеспечения возможности диагностики при возникновении
и расследовании инцидентов.

\subsubsection*{Подключаемые бизнес-правила}

Необходимо обеспечить возможность настройки функциональности системы в различных
точках сценариев нескольких прецедентов на основе заданных правил. Различная
функциональность системы должна быть изменяема в зависимости от потребностей
руководства каждого высшего учебного заведения.

\subsubsection*{Безопасность}

Все пользователи, работающие с системой, должны быть аутентифицированы. Для
различных пользователей, обладающими различными уровнями прав, необходимо
определить матрицу доступа. Все данные, хранящиеся в системе, должны быть
зашифрованы. Все действия пользователя, в частности те, которые изменяют
состояния системы, должны быть зафиксированы в системе.

\subsection*{Удобство использования}

\subsubsection*{Человеческие факторы}

Единая государственная система абитуриента нацелена на широкую аудиторию, в том
числе на людей с ограниченными возможностями. Поэтому интерфейс должен быть не
только интуитивным и понятным, но и доступным. Для этого система должна
поддерживать различные режимы работы, например, версию для слабовидящих.

Все действия пользователя, которые каким-либо образом влияют на него, должны
быть сопровождены дополнительным подтверждением (например, подача документов в
конкретный ВУЗ). Это позволит предотвратить нежелательные действия пользователя,
произошедшие по случайности.

На любое действие пользователь должен получать визуальный отклик. Ему должно
быть понятно, что система обработала действие правильно или же возникла какая-то
ошибка, которую нужно устранить.

\subsection*{Надежность}

\subsubsection*{Возможность восстановления информации}

При сбоях в работе внешних систем (например, интеграций с системами ВУЗов)
система должна сохранять всю информацию, сообщенную пользователем. После
восстановления внешней системы у пользователя не должна возникать необходимость
в повторном заполнении ранее внесенной информации.

При сбоях в работе самой системы все данные должны оставаться в исходном
состоянии. Никак из данных не должны быть повреждены. Должна быть возможность
проследить историю изменений данных и, в случае необходимости, восстановить
определенную версию данных.

\subsection*{Производительность}

Система должна выдерживать высокие нагрузки в реальном времени. Пользователи не
должны испытывать неудобства при большом спросе к системе. Около 90\% запросов,
которые делает пользователь, должны обрабатываться не более чем за 1 минуту.

\subsection*{Возможности поддержки}

\subsubsection*{Адаптация системы}

Различные ВУЗы могут иметь различные требования для абитуриентов для
поступления. Поэтому в нескольких заранее определенных точках сценария
(например, при добавлении информации о приемной компании на какое-либо
направление) нужно обеспечить возможность подключения бизнес-правил.

\subsubsection*{Конфигурирование}

Один и тот же тип информации может незначительно отличаться в зависимости от
ВУЗа. Следовательно, система должна быть настраиваемой и отражать потребности
пользователей. Этот вопрос требует тщательной дополнительной проработки,
изучения степени гибкости и способов ее достижения.

\subsection*{Бизнес-правила}

\begin{enumerate}[leftmargin=*]
  \item Каждый абитуриент должен видеть в системе на каком этапе поступления в
  тот или иной ВУЗ он находится (документы поданы, документы подтверждены,
  находится в конкурсе на каком-то месте, зачислен в ВУЗ и т. д.).
  \item Все данные пользователей должны быть надежно защищены от
  несанкционированного доступа, а также зашифрованы.
  \item Руководители приемных кампаний должны иметь возможность получать
  актуальную статистику абитуриентов, участвующих в конкурсе по конкретному
  направлению.
  \item Система должна уметь составлять отчеты о приемных кампаниях, которые
  будут доступны руководителям приемных кампаний.
  \item Все взаимодействие между абитуриентами и руководствами ВУЗов должно быть
  доступно с помощью электронных средств связи.
\end{enumerate}

\newpage

\section{Документ-концепция}

\subsection*{Дата внесения изменений}

\begin{table}[H]
  \begin{tabular}{|L{0.17\textwidth}|L{0.19\textwidth}|L{0.4\textwidth}|L{0.15\textwidth}|}
    \hline
    \textbf{Версия}                                                  &
    \textbf{Дата}                                                    &
    \textbf{Описание}                                                &
    \textbf{Автор}                                                     \\
    \hline
    Черновой начальный вариант                                       &
    31 декабря, 2023                                                 &
    Первый черновой вариант. Будет уточнен на первой стадии развития &
    Швалов Даниил                                                      \\
    \hline
  \end{tabular}
\end{table}

\subsection*{Введение}

Нам видится надежная платформа для объединения между собой абитуриентов и
руководств учебных заведений, обеспечивающая гибкую поддержку различных
бизнес-правил, а также механизмы поддержки различных интеграций с внешними
системами ВУЗов.

\subsection*{Позиционирование}

\subsubsection*{Экономические предпосылки}

Существующие по сегодняшний день платформы не позволяют абитуриентам и
руководствам учебных заведений получить пользовательский опыт, при котором
каждая из сторон сможет беспроблемно достичь своих целей. Существующие платформы
обладают очень ограниченным функционалом, не позволяют большому количеству ВУЗов
простым образом интегрироваться с системой, являются малорасширяемыми. Все это
открывает нишу для создания новой системы для создания комфортной работы между
абитуриентами и руководствами ВУЗов.

\subsubsection*{Формулировка проблемы}

Существующие платформы не обладают гибкостью и не обеспечивают интеграцию с
системами ВУЗов. Это приводит к частичной или полной невозможности создания
удобной системы, покрывающей большой спектр запросов пользователей. Эти проблемы
касаются как абитуриентов, так и руководств ВУЗов.

\subsubsection*{Место системы}

Единая государственная система абитуриента предназначена для создания удобной
платформы, на которой абитуриенты смогут найти максимально подходящий для себя
ВУЗ, а руководства учебных заведений смогут удобно и прогнозируемо понимать,
составлять отчеты и анализировать все, что связано с приемной кампанией.

\newpage

\subsection*{Заинтересованные лица}

\subsubsection*{Основные задачи высокого уровня и проблемы заинтересованных лиц}

\begin{longtable}{|L{0.15\textwidth}|L{0.12\textwidth}|L{0.45\textwidth}|L{0.15\textwidth}|}
  \hline
  \textbf{Цель высокого уровня}                                              &
  \textbf{Приоритет}                                                         &
  \textbf{Проблемы и замечания}                                              &
  \textbf{Текущие решения}                                                     \\
  \hline
  Возможность гибкой интеграции с различными системами ВУЗов                 &
  Высокий                                                                    &
  Большинство ВУЗов в данный момент имеют множество средств по автоматизации
  приемной кампании. Для того, чтобы ЕГСА имела успех среди руководств учебных
  заведений, необходимо, чтобы ЕГСА могла быть беспроблемно внедрена уже в
  существующие внешние системы. Поскольку существует множество различных систем,
  может возникнуть проблема при проектировании API. Необходимо создать такой
  интерфейс, который удовлетворял бы большому количеству существующих систем
  ВУЗов. Возможно потребуется поддерживать несколько API для различных ВУЗов &
  Существующие системы не обеспечивают беспроблемной и гибкой интеграции с
  существующими внешними системами                                             \\
  \hline
  Возможность использования ЕГСА без дополнительных систем                   &
  Средний                                                                    &
  При разработке универсальной системы будут возникать сложности в поддержке уже
  существующих сервисов. Для решения этой проблемы предлагается заранее
  проработать архитектуру системы. При этом рекомендуется использовать
  микросервисную архитектуру, при которой каждый компонент выделяется в
  микросервис. Это позволит не только сделать систему более отказоустойчивой, но
  и сделает ее более масштабируемой, а также гибкой                          &
  Существующие системы не позволяют использовать их как целостное решение для
  приемных кампаний ВУЗов                                                      \\
  \hline
\end{longtable}

\paragraph*{Задачи уровня пользователя}

\begin{itemize}[leftmargin=*]
  \item \textbf{Абитуриент}. Выбирает ВУЗ из списка по собственным критериям и
  предпочтениям. Добавляет документы и информацию о себе в систему.
  \item \textbf{Руководство приемной кампании}. Получает заявления о поступлении
  от абитуриентов, проверяет переданные документы, подтверждает участие
  абитуриента в конкурсе за место.
  \item Администрация сервиса. Занимается поддержкой сервиса, помогает
  руководствам ВУЗов в интегрировании и настройке системы.
\end{itemize}

\subsubsection*{Обзор}

\paragraph*{Перспективы продукта}

Единая государственная система абитуриентов будет использоваться большинством
ВУЗов страны. Система будет предлагать абитуриентам и руководствам ВУЗов
платформу, с помощью которой абитуриенты смогут найти подходящий ВУЗ, а
руководства ВУЗов --- прогнозировать, анализировать и проводить приемные
кампании.

\paragraph*{Преимущества системы}

\begin{longtable}{|L{0.47\textwidth}|L{0.47\textwidth}|}
  \hline
  \textbf{Свойство}                       &
  \textbf{Преимущества для заинтересованных лиц} \\
  \hline
  Система будет обеспечивать простой способ для нахождения ВУЗа, подходящего для
  абитуриента по его знаниям и интересам  &
  Абитуриент получает возможность найти и поступить в ВУЗ, который ему
  интересен. Руководство ВУЗа получает абитуриента, заинтересованного в обучении
  в данном ВУЗе                                  \\
  \hline
  Составление отчетов о приемной кампании &
  Руководству ВУЗа не придется
  самостоятельно собирать статистику и составлять отчеты о приемных кампаниях.
  Все это будет переложено на ЕГСА               \\
  \hline
\end{longtable}

\paragraph*{Основные свойства системы}

\begin{itemize}[leftmargin=*]
  \item Хранение и проверка документов абитуриента.
  \item Составление отчетов приемной кампании.
  \item Интеграция с внешними системами ВУЗов.
  \item Безопасное и надежное хранение данных.
  \item Определение и выполнение настраиваемых бизнес-правил.
  \item Отказоустойчивость, работа при высоких нагрузках.
\end{itemize}

\newpage

\section{Словарь терминов}

\subsection*{Дата внесения изменений}

\begin{table}[H]
  \begin{tabular}{|L{0.17\textwidth}|L{0.19\textwidth}|L{0.4\textwidth}|L{0.15\textwidth}|}
    \hline
    \textbf{Версия}                                                  &
    \textbf{Дата}                                                    &
    \textbf{Описание}                                                &
    \textbf{Автор}                                                     \\
    \hline
    Черновой начальный вариант                                       &
    31 декабря, 2023                                                 &
    Первый черновой вариант. Будет уточнен на первой стадии развития &
    Швалов Даниил                                                      \\
    \hline
  \end{tabular}
\end{table}

\begin{longtable}{|L{0.26\textwidth}|L{0.69\textwidth}|}
  \hline
  \textbf{Термин}                                   &
  \textbf{Определение}                                \\
  \hline
  Единая государственная система абитуриента (ЕГСА) &
  Название создаваемой системы                        \\
  \hline
  ID пользователя                                   &
  Идентификатор пользователя в формате UUID           \\
  \hline
\end{longtable}

\section{Заключение}

В ходе выполнения данного практического задания я разработал дополнительную
спецификацию, документ-концепцию и словарь терминов.

\end{document}
