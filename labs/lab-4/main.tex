\documentclass[a4paper, 14pt]{extarticle}
\usepackage[russian]{babel}
\usepackage[T1]{fontenc}
\usepackage{fontspec}
\usepackage{indentfirst}
\usepackage{enumitem}
\usepackage{graphicx}
\usepackage{scrextend}
\usepackage{longtable}
\usepackage[
  left=20mm,
  right=10mm,
  top=20mm,
  bottom=20mm
]{geometry}
\usepackage{parskip}
\usepackage{titlesec}
\usepackage{xurl}
\usepackage{hyperref}
\usepackage{float}
\usepackage[
  figurename=Рисунок,
  labelsep=endash,
]{caption}
\usepackage[outputdir=build, newfloat]{minted}
\usepackage{multirow}
\usepackage{array}

\hypersetup{
  colorlinks=true,
  linkcolor=black,
  filecolor=blue,
  urlcolor=blue,
}

\renewcommand*{\labelitemi}{---}
\setmainfont{Times New Roman}
\setmonofont{JetBrains Mono}[
  SizeFeatures={Size=11},
]

\newenvironment{code}{\captionsetup{type=listing}}{}
\SetupFloatingEnvironment{listing}{name=Листинг}

\setminted{
  fontsize=\footnotesize,
  framesep=0mm,
}

\captionsetup{width=\textwidth,justification=centering}
\captionsetup[table]{singlelinecheck=off,justification=justified}

\newcolumntype{L}[1]{>{\raggedright\let\newline\\\arraybackslash\hspace{0pt}}m{#1}}
\newcolumntype{C}[1]{>{\centering\let\newline\\\arraybackslash\hspace{0pt}}m{#1}}
\newcolumntype{R}[1]{>{\raggedleft\let\newline\\\arraybackslash\hspace{0pt}}m{#1}}

\setlength{\parskip}{6pt}

\setlength{\parindent}{1cm}
\setlist[itemize]{itemsep=0em,topsep=0em,parsep=0em,partopsep=0em,leftmargin=2.0cm}
\setlist[enumerate]{itemsep=0em,topsep=0em,parsep=0em,partopsep=0em,leftmargin=2.0cm}

\renewcommand{\thesection}{\arabic{section}.}
\renewcommand{\thesubsection}{\thesection\arabic{subsection}.}
\renewcommand{\thesubsubsection}{\thesubsection\arabic{subsubsection}.}

\titleformat{\section}{\normalfont\bfseries}{\thesection}{0.5em}{}
\titleformat{\subsection}{\normalfont\bfseries}{\thesubsection}{0.5em}{}

\titleformat*{\section}{\normalfont\bfseries}
\titleformat*{\subsection}{\normalfont\bfseries}

\linespread{1.5}
\renewcommand{\baselinestretch}{1.5}

\renewcommand{\theenumii}{(\asbuk{enumii})}
\renewcommand{\labelenumii}{\asbuk{enumii})}
\AddEnumerateCounter{\asbuk}{\@asbuk}{\cyrm}

\begin{document}

\begin{titlepage}
  \vspace{0pt plus2fill}
  \noindent

  \vspace{0pt plus6fill}
  \begin{center}
    Санкт-Петербургский национальный исследовательский университет
    информационных технологий, механики и оптики

    \vspace{0pt plus2fill}

    Практическая работа по теме:

    <<Единая государственная система абитуриента>>

    \vspace{0pt plus1fill}

    Задание №4

    <<Создание документа-концепции>>

  \end{center}

  \vspace{0pt plus7fill}
  \begin{flushright}
    Выполнил: \\
    Швалов Даниил Андреевич

    Группа: К33211

    Проверил: \\
    Осипов Никита Алексеевич
  \end{flushright}

  \vspace{0pt plus2fill}
  \begin{center}
    Санкт-Петербург

    2023
  \end{center}
\end{titlepage}

\setcounter{page}{2}

\section{Введение}

\textbf{Цель}: разработать дополнительную спецификацию, документ-концепцию и
словарь терминов.

\section{Дополнительная спецификация}

% TODO:

\section{Документ-концепция}

\subsubsection*{Введение}

% TODO:

\subsubsection*{Позиционирование}

\paragraph*{Экономические предпосылки}

% TODO:

\paragraph*{Формулировка проблемы}

% TODO:

\paragraph*{Место системы}

% TODO:

\paragraph*{Заинтересованные лица}

% TODO:

\paragraph*{Основные задачи высокого уровня и проблемы заинтересованных лиц}

% TODO:

\begin{longtable}{|L{0.225\textwidth}|L{0.225\textwidth}|L{0.225\textwidth}|L{0.225\textwidth}|}
  \hline
  \textbf{Цель высокого уровня} & \textbf{Приоритет} & \textbf{Проблемы и замечания} & \textbf{Текущие решения} \\
  \hline
  % TODO:
\end{longtable}

\paragraph*{Задачи уровня пользователя}

% TODO:

\subsubsection*{Обзор}

\paragraph*{Перспективы продукта}

% TODO:

\paragraph*{Преимущества системы}

% TODO:

\begin{longtable}{|L{0.47\textwidth}|L{0.47\textwidth}|}
  \hline
  \textbf{Свойство} & \textbf{Преимущества для заинтересованных лиц} \\
  \hline
  % TODO:
\end{longtable}

\paragraph*{Основные свойства системы}

% TODO:

\paragraph*{Другие требования и ограничения}

% TODO:

\section{Словарь терминов}

% TODO:

\section{Заключение}

В ходе выполнения данного практического задания я разработал дополнительную
спецификацию, документ-концепцию и словарь терминов.


\end{document}