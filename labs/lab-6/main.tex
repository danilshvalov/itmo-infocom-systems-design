\documentclass[a4paper, 14pt]{extarticle}
\usepackage[russian]{babel}
\usepackage[T1]{fontenc}
\usepackage{fontspec}
\usepackage{indentfirst}
\usepackage{enumitem}
\usepackage{graphicx}
\usepackage{scrextend}
\usepackage{longtable}
\usepackage[
  left=20mm,
  right=10mm,
  top=20mm,
  bottom=20mm
]{geometry}
\usepackage{parskip}
\usepackage{titlesec}
\usepackage{xurl}
\usepackage{hyperref}
\usepackage{float}
\usepackage[
  figurename=Рисунок,
  labelsep=endash,
]{caption}
\usepackage[outputdir=build, newfloat]{minted}
\usepackage{multirow}
\usepackage{array}

\hypersetup{
  colorlinks=true,
  linkcolor=black,
  filecolor=blue,
  urlcolor=blue,
}

\renewcommand*{\labelitemi}{---}
\setmainfont{Times New Roman}
\setmonofont{JetBrains Mono}[
  SizeFeatures={Size=11},
]

\newenvironment{code}{\captionsetup{type=listing}}{}
\SetupFloatingEnvironment{listing}{name=Листинг}

\setminted{
  fontsize=\footnotesize,
  framesep=0mm,
}

\captionsetup{width=\textwidth,justification=centering}
\captionsetup[table]{singlelinecheck=off,justification=justified}

\newcolumntype{L}[1]{>{\raggedright\let\newline\\\arraybackslash\hspace{0pt}}m{#1}}
\newcolumntype{C}[1]{>{\centering\let\newline\\\arraybackslash\hspace{0pt}}m{#1}}
\newcolumntype{R}[1]{>{\raggedleft\let\newline\\\arraybackslash\hspace{0pt}}m{#1}}

\setlength{\parskip}{6pt}

\setlength{\parindent}{1cm}
\setlist[itemize]{itemsep=0em,topsep=0em,parsep=0em,partopsep=0em,leftmargin=2.0cm}
\setlist[enumerate]{itemsep=0em,topsep=0em,parsep=0em,partopsep=0em,leftmargin=2.0cm}

\renewcommand{\thesection}{\arabic{section}.}
\renewcommand{\thesubsection}{\thesection\arabic{subsection}.}
\renewcommand{\thesubsubsection}{\thesubsection\arabic{subsubsection}.}

\titleformat{\section}{\normalfont\bfseries}{\thesection}{0.5em}{}
\titleformat{\subsection}{\normalfont\bfseries}{\thesubsection}{0.5em}{}

\titleformat*{\section}{\normalfont\bfseries}
\titleformat*{\subsection}{\normalfont\bfseries}

\linespread{1.5}
\renewcommand{\baselinestretch}{1.5}

\renewcommand{\theenumii}{(\asbuk{enumii})}
\renewcommand{\labelenumii}{\asbuk{enumii})}
\AddEnumerateCounter{\asbuk}{\@asbuk}{\cyrm}

\begin{document}

\begin{titlepage}
  \vspace{0pt plus2fill}
  \noindent

  \vspace{0pt plus6fill}
  \begin{center}
    Санкт-Петербургский национальный исследовательский университет
    информационных технологий, механики и оптики

    \vspace{0pt plus2fill}

    Практическая работа по теме:

    <<Разработка спецификации требований к программному обеспечению>>

    \vspace{0pt plus1fill}

    Задание №6

    <<Разработка технического задания>>

  \end{center}

  \vspace{0pt plus7fill}
  \begin{flushright}
    Выполнил: \\
    Швалов Даниил Андреевич

    Группа: К33211

    Проверил: \\
    Осипов Никита Алексеевич
  \end{flushright}

  \vspace{0pt plus2fill}
  \begin{center}
    Санкт-Петербург

    2023
  \end{center}
\end{titlepage}

\setcounter{page}{2}

\section{Общие сведения}

\subsection{Наименование системы}

\subsubsection{Полное наименование системы}

Полное наименование: Единая государственная система абитуриента

\subsubsection{Краткое наименование системы}

Краткое наименование: ЕГСА

\subsection{Плановые сроки начала и окончания работы}

Все работы по проектированию и разработке ЕГСА передаются Разработчиком поэтапно
в соответствии с календарным планом Проекта. По окончании каждого из этапов
работ Разработчик сдает Заказчику соответствующие отчётные документы этапа.

\section{Назначение и цели создания системы}

\subsection{Назначение системы}

Единая государственная система абитуриента предназначена для упрощения
проведения приемных кампаний, как для абитуриентов, так и для руководств учебных
заведений. Основным назначением ЕГСА является автоматизация
информационно-аналитической деятельности в бизнес-процессах Заказчика. В рамках
проекта она автоматизируется в следующих бизнес-процессах:
\begin{enumerate}
  \item подача, обработка и подтверждение документов абитуриента;
  \item отслеживание статуса поступления абитуриентов;
  \item составление отчетов о проведении приемной кампании для руководств ВУЗов;
  \item предоставление информации о каждом направлении в ВУЗе, о ВУЗе в целом.
\end{enumerate}

\subsection{Цели создания системы}

Единая государственная система абитуриента создается с целью:
\begin{itemize}
  \item упрощения и оптимизации процесса поступления, предоставляя возможность
  проводить всю приемную кампанию онлайн через Интернет, без необходимости
  очного взаимодействия;
  \item повышения эффективности и прозрачности приемной кампании за счет
  автоматизации процессов составления отчетов и аналитики.
\end{itemize}

В результате создания государственной системы должны быть улучшены значения
следующих показателей:
\begin{itemize}
  \item время, затрачиваемое абитуриентом на выбор ВУЗа, подачу документов,
  получения приказа о зачислении;
  \item время, затрачиваемое на сбор и анализ информации, связанной с приемной
  кампанией.
\end{itemize}

\section{Характеристика объектов автоматизации}

\begin{longtable}{|L{0.3\textwidth}|L{0.3\textwidth}|L{0.3\textwidth}|}
  \hline
  \textbf{Наименование процесса}                             &
  \textbf{Возможность автоматизации}                         &
  \textbf{Решение об автоматизации в ходе проекта}             \\
  \hline
  Внесение информации о достижениях и документах абитуриента &
  Возможна                                                   &
  Будет автоматизирован                                        \\
  \hline
  Составление отчетов об итогах проведения приемной кампании &
  Возможна                                                   &
  Будет автоматизирован                                        \\
  \hline
\end{longtable}

\section{Требования к системе}

\subsection{Требования к системе в целом}

\subsubsection{Требование к структуре и функционированию системы}

Система ЕГСА должна являться централизованной: все данные, в том числе
персональные данные, должны располагаться в едином и защищенном хранилище.

В ЕГСА предполагается выделить следующие функциональные подсистемы:
\begin{itemize}
  \item подсистема персональных хранения данных, к которым предъявляются
  повышенные требования безопасности;
  \item подсистема хранения системных данных, которые нужны для работы системы.
  (метрики, логи и т. п.);
  \item подсистема хранения прочей информации о пользователе, не относящейся к
  персональным данным;
  \item подсистема анализа и обработки данных, составления отчетов.
\end{itemize}

В качестве протокола взаимодействия между компонентами Системы на
транспортно-сетевом уровне необходимо использовать протокол TCP/IP, поскольку в
случае системы потери данных не приемлемы.

Для организации информационного обмена между компонентами Системы должны
использоваться специальные протоколы прикладного уровня, такие как: HTTP и его
расширение HTTPS.

Система должна поддерживать следующие режимы функционирования:
\begin{itemize}
  \item основной режим функционирования, при котором система выполняет все свои
  основные функции;
  \item профилактический режим функционирования, при котором в системе можно
  выявлять неисправности и ошибки в функционировании.
\end{itemize}

В основном режиме функционирования ЕГСА должна обеспечивать:
\begin{itemize}
  \item работу пользователей в режиме --- 24 часов в день, 7 дней в неделю;
  \item выполнение всех своих функций: сбор, обработка и анализ пользовательских
  данных.
\end{itemize}

В профилактическом режиме Система должна иметь возможность производить следующие
работы:
\begin{itemize}
  \item техническое обслуживание;
  \item устранение аварийных ситуаций.
\end{itemize}

Каждая подсистема всей Системы должна обеспечивать требования по диагностике ее
состояния для повышения надежности функционирования всей Системы. Подсистемы и
Система в целом должны вести журналы инцидентов для возможности диагностики при
возникновении внештатных ситуаций.

\subsubsection{Требования к численности и квалификации персонала
  системы и режиму его работы}

В состав персонала, необходимого для обеспечения работы Системы в рамках
соответствующих подразделений Заказчика, необходимы следующие ответственные
лица:
\begin{itemize}
  \item руководитель подразделения (1 человек);
  \item администратор подсистемы хранения (5 человек);
  \item администратор подсистемы обработки и анализа данных (4 человека);
  \item оператор службы поддержки (10 человек).
\end{itemize}

Данные лица должны выполнять следующие функциональные обязанности:
\begin{itemize}
  \item руководитель подразделения: обеспечивает общее руководство группой
  сопровождение;
  \item администратор подсистемы хранения: обеспечивает контроль за
  безопасностью и надежность хранимых данных;
  \item администратор обработки и анализа данных: обеспечивает контроль
  процессов обработки и анализа данных;
  \item оператор службы поддержки: обеспечивает поддержку пользователей и
  внешних разработчиков.
\end{itemize}

К квалификации персонала, эксплуатирующего ЕГСА, предъявляются следующие
требования:
\begin{itemize}
  \item руководитель подразделения: навыки управления командами, работающими с
  отказоустойчивыми системами, работающими в реальном времени;
  \item администратор подсистемы хранения: углубленные знания работы СУБД,
  понимание принципов безопасного хранения данных, умения по работе с
  отказоустойчивыми системами;
  \item администратор обработки и анализа данных: понимание принципов обработки
  и анализа данных;
  \item оператор службы поддержки: умение общаться с пользователем.
\end{itemize}

Персонал, работающий с Системой и выполняющий функции её сопровождения и
обслуживания, должен работать в следующих режимах:
\begin{itemize}
  \item руководитель подразделения: в соответствии с основным рабочим графиком;
  \item администратор подсистемы хранения: в две смены, поочередно;
  \item администратор обработки и анализа данных: в две смены, поочередно;
  \item оператор службы поддержки: в две смены, поочередно.
\end{itemize}

\subsubsection{Показатели назначения}

Обеспечение приспособляемости системы должно выполняться за счет:
\begin{itemize}
  \item своевременного администрирования ЕГСА;
  \item постоянного улучшения и модернизации процессов сбора, обработки и
  анализа данных в соответствии с текущими требованиями;
  \item адаптации к новым потребностям пользователей.
\end{itemize}

\subsubsection{Требования к эргономике и технической эстетике}

Подсистема обработки и анализа данных должна предоставлять пользователям
понятный и удобный интерфейс, который отвечает следующим требованиям:
\begin{itemize}
  \item интерфейсы подсистем типизированы;
  \item наличие локализованного языкового интерфейса;
  \item использование стандартных шрифтов, доступных на большинстве устройств;
  \item наличие горячих клавиш;
  \item прозрачная работа системы с информированием о статусе выполнения той или
  иной функции.
\end{itemize}

\subsubsection{Требования к защите информации от несанкционированного доступа}

Обеспечение информационное безопасности ЕГСА должно удовлетворять следующим требованиям:
\begin{itemize}
  \item защита должна обеспечиваться комплексом проверенных сертифицированных
  программно-технических средств и поддерживающих их организационных мер.
  \item защита должна обеспечиваться на всех этапах сбора и обработки
  информации, равно как и во всех режимах.
  \item средства защиты не должны значимо влиять на основные функциональные
  характеристики (такие как надежность и производительность).
\end{itemize}

Средства антивирусной защиты должны быть установлены на всех рабочих местах
пользователей, администраторов и операторов ЕГСА. Средства антивирусной защиты
рабочих местах пользователей и администраторов должны обеспечивать:
\begin{itemize}
  \item сканирование, удаление вирусов и протоколирование вирусной активности;
  \item автоматическое обновление вирусных сигнатур и баз данных.
\end{itemize}

\subsection{Требования к функциям, выполняемым системой}

\subsubsection{Подсистема сбора, обработки и загрузки данных}

\begin{longtable}{|L{0.35\textwidth}|L{0.61\textwidth}|}
  \hline
  \textbf{Функция}                                               &
  \textbf{Задача}                                                     \\
  \hline
  Управляет процессами сбора, обработки и загрузки данных        &
  Изменение, формирование последовательности выполнения, определение и изменение
  расписания процессов сбора, обработки и загрузки данных             \\
  \hline
  Выполнение процессов сбора, обработки и загрузки данных из различных
  источников                                                     &
  Запуск процедур сбора данных из систем источников, загрузка данных в область
  постоянного хранения, обработка и преобразование извлечённых данных \\
  \hline
  Протоколирование результаты сбора, обработки и загрузки данных &
  Ведение журналов результатов сбора, обработки и загрузки данных, оперативное
  извещение пользователей о всех нештатных ситуациях в процессе работы
  подсистемы                                                          \\
  \hline
\end{longtable}

\subsubsection{Временной регламент реализации каждой функции, задачи}

\begin{longtable}{|L{0.48\textwidth}|L{0.48\textwidth}|}
  \hline
  \textbf{Задача}                                                     &
  \textbf{Требования к временному регламенту}                              \\
  \hline
  Изменение, формирование последовательности выполнения, определение и изменение
  расписания процессов сбора, обработки и загрузки данных             &
  Весь период функционирования системы, при возникновении необходимости
  изменения процессов сбора, обработки и загрузки данных                   \\
  \hline
  Запуск процедур сбора данных из систем источников, загрузка данных в область
  постоянного хранения, обработка и преобразование извлечённых данных &
  После готовности данных, ежедневно во временном интервале 00:30 -- 05:00 \\
  \hline
  Ведение журналов результатов сбора, обработки и загрузки данных, оперативное
  извещение пользователей о всех нештатных ситуациях в процессе работы
  подсистемы                                                          &
  Регулярно, при возникновении нештатной ситуации в процессе работы подсистемы
  \\
  \hline
\end{longtable}

\subsection{Требования к видам обеспечения}

\subsubsection{Требования к информационному обеспечению}

Для реализации подсистемы хранения данных должна использоваться промышленная
СУБД PostgreSQL.

К контролю данных предъявляются следующие требования:
\begin{itemize}
  \item система должна протоколировать все события, связанные с изменением
  состояния, и иметь возможность в случае сбоя в работе восстанавливать свое
  состояние, используя ранее запротоколированные изменения данных.
\end{itemize}

К хранению данных предъявляются следующие требования:
\begin{itemize}
  \item хранение исторических данных в системе должно производиться не более чем
  за два предыдущих года. По истечению данного срока данные должны переходить в
  архивное хранилище;
  \item исторические данные, превышающие двухлетний порог, должны храниться на
  ленточном массиве с возможностью их восстановления;
\end{itemize}

К обновлению и восстановлению данных предъявляются следующие требования:
\begin{itemize}
  \item для всех подсистем необходимо обеспечить резервное копирование файлов не
  реже чем раз в 2 недели и хранение копии на протяжении 2-х месяцев;
\end{itemize}

\subsubsection{Требования к лингвистическому обеспечению}

При реализации ЕГСА необходимо применять следующие языки программирования: C++,
HTML, CSS, TypeScript и другие, если это требуется.

Для организации диалога системы с пользователем должен применяться графический
оконный пользовательский интерфейс.

\section{Состав и содержание работ по созданию системы}

Работы по созданию системы выполняются в четыре этапа:
\begin{enumerate}
  \item проектирование архитектуры системы (1 месяц);
  \item разработка технического проекта (3 месяца);
  \item разработка документации (1 месяц);
  \item введение в эксплуатацию (1 месяц).
\end{enumerate}

\end{document}
