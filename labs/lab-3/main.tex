\documentclass[a4paper, 14pt]{extarticle}
\usepackage[russian]{babel}
\usepackage[T1]{fontenc}
\usepackage{fontspec}
\usepackage{indentfirst}
\usepackage{enumitem}
\usepackage{graphicx}
\usepackage{scrextend}
\usepackage{longtable}
\usepackage[
  left=20mm,
  right=10mm,
  top=20mm,
  bottom=20mm
]{geometry}
\usepackage{parskip}
\usepackage{titlesec}
\usepackage{xurl}
\usepackage{hyperref}
\usepackage{float}
\usepackage[
  figurename=Рисунок,
  labelsep=endash,
]{caption}
\usepackage[outputdir=build, newfloat]{minted}
\usepackage{multirow}
\usepackage{array}

\hypersetup{
  colorlinks=true,
  linkcolor=black,
  filecolor=blue,
  urlcolor=blue,
}

\renewcommand*{\labelitemi}{---}
\setmainfont{Times New Roman}
\setmonofont{JetBrains Mono}[
  SizeFeatures={Size=11},
]

\newenvironment{code}{\captionsetup{type=listing}}{}
\SetupFloatingEnvironment{listing}{name=Листинг}

\setminted{
  fontsize=\footnotesize,
  framesep=0mm,
}

\captionsetup{width=\textwidth,justification=centering}
\captionsetup[table]{singlelinecheck=off,justification=justified}

\newcolumntype{L}[1]{>{\raggedright\let\newline\\\arraybackslash\hspace{0pt}}m{#1}}
\newcolumntype{C}[1]{>{\centering\let\newline\\\arraybackslash\hspace{0pt}}m{#1}}
\newcolumntype{R}[1]{>{\raggedleft\let\newline\\\arraybackslash\hspace{0pt}}m{#1}}

\setlength{\parskip}{6pt}

\setlength{\parindent}{1cm}
\setlist[itemize]{itemsep=0em,topsep=0em,parsep=0em,partopsep=0em,leftmargin=2.0cm}
\setlist[enumerate]{itemsep=0em,topsep=0em,parsep=0em,partopsep=0em,leftmargin=2.0cm}

\renewcommand{\thesection}{\arabic{section}.}
\renewcommand{\thesubsection}{\thesection\arabic{subsection}.}
\renewcommand{\thesubsubsection}{\thesubsection\arabic{subsubsection}.}

\titleformat{\section}{\normalfont\bfseries}{\thesection}{0.5em}{}
\titleformat{\subsection}{\normalfont\bfseries}{\thesubsection}{0.5em}{}

\titleformat*{\section}{\normalfont\bfseries}
\titleformat*{\subsection}{\normalfont\bfseries}

\linespread{1.5}
\renewcommand{\baselinestretch}{1.5}

\renewcommand{\theenumii}{(\asbuk{enumii})}
\renewcommand{\labelenumii}{\asbuk{enumii})}
\AddEnumerateCounter{\asbuk}{\@asbuk}{\cyrm}

\begin{document}

\begin{titlepage}
  \vspace{0pt plus2fill}
  \noindent

  \vspace{0pt plus6fill}
  \begin{center}
    Санкт-Петербургский национальный исследовательский университет
    информационных технологий, механики и оптики

    \vspace{0pt plus2fill}

    Практическая работа по теме:

    <<Единая государственная система абитуриента>>

    \vspace{0pt plus1fill}

    Задание №3

    <<Построение и исследование моделей сценария использования (Use Case)>>

  \end{center}

  \vspace{0pt plus7fill}
  \begin{flushright}
    Выполнил: \\
    Швалов Даниил Андреевич

    Группа: К33211

    Проверил: \\
    Осипов Никита Алексеевич
  \end{flushright}

  \vspace{0pt plus2fill}
  \begin{center}
    Санкт-Петербург

    2023
  \end{center}
\end{titlepage}

\setcounter{page}{2}

\section{Введение}

\textbf{Цель}: для выбранного варианта инфокоммуникационной системы реализовать
развернутое описание основного прецедента, на основании которых будет
организована реализация проекта. Разработать развернутое описание основного
прецедента.

\section{Ход работы}

Для единой государственной системы абитуриента было создано следующее
развернутое описание основного прецедента.

\subsubsection*{Прецедент. Подача документов на поступление в ВУЗ}

{
  \setlength{\parindent}{0pt}

  \textbf{Основной исполнитель}. Абитуриент.

  \textbf{Заинтересованные лица и их требования}
  \begin{itemize}
    \item Абитуриент. Хочет быстро и просто внести в систему свои документы,
    выбрать ВУЗ по его знаниям и интересам, подать заявление и увидеть свои
    шансы на поступления.
    \item Руководство учебного заведения. Хочет видеть список всех поступающих
    абитуриентов, иметь систему, автоматизирующую задачи приемной кампании.
  \end{itemize}

  \textbf{Предусловия}. Абитуриент идентифицирован и аутентифицирован в системе.

  \textbf{Результаты (постусловия)}. Данные о подаче заявления сохранены.
  Документы, предоставленные абитуриентом, также сохранены, защищены и
  направлены в ВУЗ, в который абитуриент подал документы. Информация о
  количестве поданных заявлений на данное направление в данном ВУЗе обновлена.
  Рейтинг абитуриентов пересчитан.

  \textbf{Основной успешный сценарий (или основной процесс)}:
  \begin{enumerate}
    \item абитуриент заходит на сайт и авторизуется в системе;
    \item абитуриент загружает все необходимые документы о себе: аттестат,
    информация личных достижений и т. п.;
    \item система сохраняет информацию о абитуриенте;
    \item абитуриент переходит на страницу со списком ВУЗов, фильтрует
    ВУЗы по интересам, направлениям, потребностям, знаниям и отзывам студентов;
    \item абитуриент выбирает один или несколько ВУЗов (до 3 шт.), а также
    направления (также до 3 шт.), куда он хотел бы поступить;
    \item абитуриент подписывает заявление на поступление в данные ВУЗы;
    \item система обрабатывает заявления от абитуриента, отправляет информацию
    руководствам учебных заведений;
    \item руководства учебных заведений подтверждают прием документов и
    заявления от абитуриента;
    \item система добавляет абитуриента в очередь на поступление в ВУЗ.
  \end{enumerate}

  \textbf{Расширения (или альтернативные потоки)}:
  \begin{enumerate}
    \item[2а.] При неправильной загрузке документов абитуриента (неправильно сделана
      фотография, не загружены подтверждения личных достижений и т. п.) или при
      дополнительных требованиях ВУЗа.
      \begin{enumerate}
        \item абитуриент догружает все недостающие или неправильно сделанные
        документы и информацию о себе;
        \item руководство учебного заведения перепроверяет новую загруженную
        информацию.
      \end{enumerate}
    \item[3а.] Ошибка при сохранении информации об абитуриенте.
      \begin{enumerate}
        \item система уведомляет об ошибке пользователя и переходит в предыдущее
        состояние;
        \item пользователь пытается снова загрузить документы.
      \end{enumerate}
    \item[6-8а.] Абитуриент хочет отозвать заявление из ВУЗа.
      \begin{enumerate}
        \item абитуриент выбирает заявление, которое нужно отменить;
        \item система получает информацию об отмене заявления, направляет
        уведомление руководству учебного заведения;
        \item руководство учебного заведения фиксирует отмену заявления;
        \item система обновляет список абитуриентов, подавших заявление в ВУЗ.
      \end{enumerate}
  \end{enumerate}

  \textbf{Специальные требования}:
  \begin{itemize}
    \item все документы и информация о абитуриентах должны быть защищены и
    храниться в зашифрованном виде;
    \item система должна выдерживать высокие нагрузки: использование системы
    большим количеством пользователей единовременно не должно препятствовать
    работе системы;
    \item при сбое системы никакие данные не должны быть повреждены, все
    операции, выполняемые во время сбоя, должны быть отменены без повреждения
    данных.
  \end{itemize}

  \textbf{Список технологий и типов данных}:
  \begin{itemize}
    \item доступ систему должен быть возможен со всех основных типов устройств и
    операционных систем (iOS, Android, Windows, Linux, macOS);
    \item база данных, используемая для хранения данных об абитуриентах, должна
    соответствовать требованиям ACID, а также обладать механизмом транзакций.
  \end{itemize}

  \textbf{Частота использования}:
  \begin{itemize}
    \item для абитуриентов: от 1 до 5 раз за все время;
    \item для руководств учебных заведений: ежегодно во время работы приемной
    кампании.
  \end{itemize}

  \textbf{Открытые вопросы}:
  \begin{itemize}
    \item изучить нормативные документы по организации процессов приемных
    кампаний разных ВУЗов;
    \item исследовать максимально возможную нагрузку в пик подачи заявлений во
    время приемной кампании;
    \item выяснить, какое количество данных будет хранить система.
  \end{itemize}
}

\section{Заключение}

В ходе выполнения данного практического задания я реализовал развернутое
описание основного прецедента.

\end{document}