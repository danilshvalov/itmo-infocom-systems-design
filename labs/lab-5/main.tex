\documentclass[a4paper, 14pt]{extarticle}
\usepackage[russian]{babel}
\usepackage[T1]{fontenc}
\usepackage{fontspec}
\usepackage{indentfirst}
\usepackage{enumitem}
\usepackage{graphicx}
\usepackage{scrextend}
\usepackage{longtable}
\usepackage[
  left=20mm,
  right=10mm,
  top=20mm,
  bottom=20mm
]{geometry}
\usepackage{parskip}
\usepackage{titlesec}
\usepackage{xurl}
\usepackage{hyperref}
\usepackage{float}
\usepackage[
  figurename=Рисунок,
  labelsep=endash,
]{caption}
\usepackage[outputdir=build, newfloat]{minted}
\usepackage{multirow}
\usepackage{array}

\hypersetup{
  colorlinks=true,
  linkcolor=black,
  filecolor=blue,
  urlcolor=blue,
}

\renewcommand*{\labelitemi}{---}
\setmainfont{Times New Roman}
\setmonofont{JetBrains Mono}[
  SizeFeatures={Size=11},
]

\newenvironment{code}{\captionsetup{type=listing}}{}
\SetupFloatingEnvironment{listing}{name=Листинг}

\setminted{
  fontsize=\footnotesize,
  framesep=0mm,
}

\captionsetup{width=\textwidth,justification=centering}
\captionsetup[table]{singlelinecheck=off,justification=justified}

\newcolumntype{L}[1]{>{\raggedright\let\newline\\\arraybackslash\hspace{0pt}}m{#1}}
\newcolumntype{C}[1]{>{\centering\let\newline\\\arraybackslash\hspace{0pt}}m{#1}}
\newcolumntype{R}[1]{>{\raggedleft\let\newline\\\arraybackslash\hspace{0pt}}m{#1}}

\setlength{\parskip}{6pt}

\setlength{\parindent}{1cm}
\setlist[itemize]{itemsep=0em,topsep=0em,parsep=0em,partopsep=0em,leftmargin=2.0cm}
\setlist[enumerate]{itemsep=0em,topsep=0em,parsep=0em,partopsep=0em,leftmargin=2.0cm}

\renewcommand{\thesection}{\arabic{section}.}
\renewcommand{\thesubsection}{\thesection\arabic{subsection}.}
\renewcommand{\thesubsubsection}{\thesubsection\arabic{subsubsection}.}

\titleformat{\section}{\normalfont\bfseries}{\thesection}{0.5em}{}
\titleformat{\subsection}{\normalfont\bfseries}{\thesubsection}{0.5em}{}

\titleformat*{\section}{\normalfont\bfseries}
\titleformat*{\subsection}{\normalfont\bfseries}

\linespread{1.5}
\renewcommand{\baselinestretch}{1.5}

\renewcommand{\theenumii}{(\asbuk{enumii})}
\renewcommand{\labelenumii}{\asbuk{enumii})}
\AddEnumerateCounter{\asbuk}{\@asbuk}{\cyrm}

\begin{document}

\begin{titlepage}
  \vspace{0pt plus2fill}
  \noindent

  \vspace{0pt plus6fill}
  \begin{center}
    Санкт-Петербургский национальный исследовательский университет
    информационных технологий, механики и оптики

    \vspace{0pt plus2fill}

    Практическая работа по теме:

    <<Единая государственная система абитуриента>>

    \vspace{0pt plus1fill}

    Задание №5

    <<Разработка спецификации требований к программному обеспечению>>

  \end{center}

  \vspace{0pt plus7fill}
  \begin{flushright}
    Выполнил: \\
    Швалов Даниил Андреевич

    Группа: К33211

    Проверил: \\
    Осипов Никита Алексеевич
  \end{flushright}

  \vspace{0pt plus2fill}
  \begin{center}
    Санкт-Петербург

    2023
  \end{center}
\end{titlepage}

\setcounter{page}{2}

\section{Цель}

В наши дни процесс поступления сильно упростился по сравнению с предыдущими
десятилетиями. Однако до сих пор поступление нельзя назвать приятным процессом.
Он не только связан с большим стрессом для абитуриентов, но и с большим
количеством повторяющейся из года в год однотипной работы для руководителей
приемной кампании ВУЗов. Единая государственная система абитуриента призвана
избавить от проблем, которые проявляются при текущем формате приемных кампаний,
и упростить процесс поступления для обеих сторон. С помощью единой платформы
абитуриенты смогут находить ВУЗ, который интересен им и подходит по их знаниям и
умениям, а руководители ВУЗов смогут избавиться от большого количества
однотипных задач, автоматизировав их.

\subsection{Масштаб}

Единая государственная система абитуриента --- это платформа, которая позволяет
абитуриентам с помощью Интернета подавать заявление на поступление в ВУЗы страны
в режиме реального времени. Также система избавляет абитуриентов от личного
присутствия, позволяет узнать полную информацию о ВУЗе и упросить всю работу с
документами, которые должны быть переданы в учебное заведение. Все это применимо
и к руководствам ВУЗов, которые также заинтересованы в поступлении абитуриентов.

\section{Краткая характеристика модели прецедентов}

Как абитуриент:
\begin{itemize}
  \item я хочу видеть требования, которые выставляет ВУЗ, для поступления, чтобы
  понимать, что я и мои знания соответствуют требованиям ВУЗа;
  \item я хочу знать, какие индивидуальные достижения учитываются при
  поступлении в данный ВУЗ, чтобы понимать, сколько итоговых баллов у меня будет
  при поступлении;
  \item я хочу знать, какое количество мест предоставляет ВУЗ для данного
  направления, чтобы понимать, смогу ли я попасть на это направление;
  \item я хочу видеть интерактивный список студентов, которые подали документы
  на это направление в данный ВУЗ, чтобы понимать, хватит ли мне баллов для
  прохождения на данное направление;
  \item я хочу видеть олимпиады, которые принимает ВУЗ, чтобы понимать, могу ли
  я поступить без вступительных испытаний;
  \item я хочу знать, требуется ли сдавать дополнительные вступительные
  испытания, чтобы понимать, нужно ли готовиться к дополнительным экзаменам
  перед поступлением;
  \item я хочу видеть, предоставляемые ВУЗом условия, такие как общежитие,
  военная кафедра и т. п., чтобы понимать, что ВУЗ удовлетворяет моим
  пожеланиям;
  \item я хочу видеть отзывы от реальных студентов о процессах обучения в данном
  ВУЗе, чтобы иметь понимание о реальном качестве обучения в этом ВУЗе;
  \item я хочу иметь возможность удаленно подавать заявление на зачисление в
  данный ВУЗ, чтобы не ехать в ВУЗ лично, а делать все через Интернет.
\end{itemize}

Как руководство учебного заведения:
\begin{itemize}
  \item я хочу видеть количество студентов, которое собирается поступить на
  данное направление, чтобы понимать примерное количество будущих студентов;
  \item я хочу предоставлять наиболее актуальную информацию для каждого
  направления и о ВУЗе в целом, чтобы абитуриенты имели более правдоподобную
  картину о ВУЗе;
  \item я хочу иметь возможность принимать заявления на поступление от
  абитуриентов в удаленном формате, чтобы упростить процесс приемной кампании;
  \item я хочу иметь возможность получать отчетность по приемным кампаниям,
  чтобы упростить процесс обработки и составления документов и отчетов по
  приемным кампаниям.
\end{itemize}

\section{Характеристика акторов}

\textbf{Абитуриент}. Выбирает ВУЗ из списка по собственным критериям и
предпочтениям. Добавляет документы и информацию о себе в систему.

\textbf{Руководство учебного заведения}. Получает заявления о поступлении от
абитуриентов, проверяет переданные документы, подтверждает участие абитуриента в
конкурсе за место.

\textbf{Администрация сервиса}. Занимается поддержкой сервиса, помогает
руководствам ВУЗов в интегрировании и настройке системы.

\section{Требования}

\subsection{Функциональные требования}

\begin{enumerate}[leftmargin=*]
  \item При изменениях в системе (например, изменение статуса абитуриента в
  приемной кампании) все пользователи, которых затрагивает это изменение, должны
  быть уведомлены различными средствами связи (push-уведомления, почтовая
  рассылка и т. п.).
  \item У пользователей должна быть возможность проследить историю изменений
  состояния, например, для абитуриента таковым может быть место в конкурсе в
  зависимости от времени.
  \item Абитуриенты должны иметь возможность загружать документы, которые в
  последствии будут переданы руководству учебного заведения.
  \item Вся информация, которая относится к персональным данным, должна быть
  зашифрована и должна быть доступна лишь ограниченному кругу лиц.
\end{enumerate}

\subsection{Нефункциональные требования}

\begin{enumerate}[leftmargin=*]
  \item Система должна выдерживать высокие нагрузки, при этом пользователи не
  должны замечать проблемы в работе сервиса при пиковых нагрузках. Все данные
  при критических нагрузках не должны быть повреждены.
  \item Система должна быть расширяемой и гибкой настолько, чтобы у руководств
  ВУЗ была возможность присоединить внешнюю систему ВУЗа к ЕГСА.
  \item Доступ к системе должен быть возможен со всех типов популярных
  устройств, таких как мобильные телефоны на различных операционных системах
  (iOS и Android), с персонального компьютера (Linux, Windows, macOS).
\end{enumerate}


\section{
  Требования к интерактивной документации пользователя в системе подсказок
}

\begin{enumerate}[leftmargin=*]
  \item В системе должен существовать раздел со справочной информацией, в
  которой пользователи смогут узнать о функциях системы. Также должна
  существовать документация для разработчиков, в которой будет описан процесс
  интеграции с внешними системами. 
  \item Дизайн платформы должен быть интуитивным и понятным. Все места, в
  которых назначение кнопки или поля могут быть восприняты неоднозначно, должны
  быть сделаны вместе со всплывающей подсказкой или пояснением.
  \item В системе должна быть возможность переключения языка.
\end{enumerate}

\section{Ограничения проектирования}

\subsection*{Backend}

Разработка backend планируется на C++, поскольку система должна выдерживать
высокие нагрузки, должна быть производительной и стабильной. Под C++ существует
множество библиотек и фреймворков, проверенных временем, которые позволят
создать систему, отвечающей требованиям производительности и безопасности.

\subsection*{Frontend}

В качестве фреймворка для frontend планируется использовать фреймворк React
совместно с языком TypeScript. Система будет обладать большим количеством
функций и компонентов, поэтому рациональным решением является использование
популярного фреймворка React, поддерживающего различные парадигмы создания
графических компонентов.

\subsection*{База данных}

В качестве базы данных была выбрана СУБД PostgreSQL. Это свободная
объектно-реляционная система управления базами данных, которая обладает богатым
функционалом, проверена временем и хорошо себя зарекомендовавшая.

\subsection*{Тестирование и процесс разработки}

Для разработки и тестирования планируется использовать систему GitLab, которая
позволяет настроить непрерывную интеграцию и тестирование (CI/CD), обладает
большим количеством различных функций для совместной разработки, код-ревью и
хранения кода.

\section{Замечания, касающиеся законности, авторских прав и т.д}

Система должна соответствовать всем соответствующим законам и постановлениям,
включая законы о конфиденциальности данных, права интеллектуальной собственности
и другие соответствующие правовые положения.

\newpage

\section{Глоссарий}

\begin{longtable}{|L{0.26\textwidth}|L{0.69\textwidth}|}
  \hline
  \textbf{Термин}                                   &
  \textbf{Определение}                                \\
  \hline
  Единая государственная система абитуриента (ЕГСА) &
  Название создаваемой системы                        \\
  \hline
  ID пользователя                                   &
  Идентификатор пользователя в формате UUID           \\
  \hline
\end{longtable}

\newpage

\section*{Приложение. Спецификации прецедентов}

\subsection*{Прецедент. Подача документов на поступление в ВУЗ}

{
  \setlength{\parindent}{0pt}

  \textbf{Основной исполнитель}. Абитуриент.

  \textbf{Заинтересованные лица и их требования}
  \begin{itemize}
    \item Абитуриент. Хочет быстро и просто внести в систему свои документы,
    выбрать ВУЗ по его знаниям и интересам, подать заявление и увидеть свои
    шансы на поступления.
    \item Руководство учебного заведения. Хочет видеть список всех поступающих
    абитуриентов, иметь систему, автоматизирующую задачи приемной кампании.
  \end{itemize}

  \textbf{Предусловия}. Абитуриент идентифицирован и аутентифицирован в системе.

  \textbf{Результаты (постусловия)}. Данные о подаче заявления сохранены.
  Документы, предоставленные абитуриентом, также сохранены, защищены и
  направлены в ВУЗ, в который абитуриент подал документы. Информация о
  количестве поданных заявлений на данное направление в данном ВУЗе обновлена.
  Рейтинг абитуриентов пересчитан.

  \textbf{Основной успешный сценарий (или основной процесс)}:
  \begin{enumerate}
    \item абитуриент заходит на сайт и авторизуется в системе;
    \item абитуриент загружает все необходимые документы о себе: аттестат,
    информация личных достижений и т. п.;
    \item система сохраняет информацию о абитуриенте;
    \item абитуриент переходит на страницу со списком ВУЗов, фильтрует
    ВУЗы по интересам, направлениям, потребностям, знаниям и отзывам студентов;
    \item абитуриент выбирает один или несколько ВУЗов (до 3 шт.), а также
    направления (также до 3 шт.), куда он хотел бы поступить;
    \item абитуриент подписывает заявление на поступление в данные ВУЗы;
    \item система обрабатывает заявления от абитуриента, отправляет информацию
    руководствам учебных заведений;
    \item руководства учебных заведений подтверждают прием документов и
    заявления от абитуриента;
    \item система добавляет абитуриента в очередь на поступление в ВУЗ.
  \end{enumerate}

  \textbf{Расширения (или альтернативные потоки)}:
  \begin{enumerate}
    \item[2а.] При неправильной загрузке документов абитуриента (неправильно сделана
      фотография, не загружены подтверждения личных достижений и т. п.) или при
      дополнительных требованиях ВУЗа.
      \begin{enumerate}
        \item абитуриент догружает все недостающие или неправильно сделанные
        документы и информацию о себе;
        \item руководство учебного заведения перепроверяет новую загруженную
        информацию.
      \end{enumerate}
    \item[3а.] Ошибка при сохранении информации об абитуриенте.
      \begin{enumerate}
        \item система уведомляет об ошибке пользователя и переходит в предыдущее
        состояние;
        \item пользователь пытается снова загрузить документы.
      \end{enumerate}
    \item[6-8а.] Абитуриент хочет отозвать заявление из ВУЗа.
      \begin{enumerate}
        \item абитуриент выбирает заявление, которое нужно отменить;
        \item система получает информацию об отмене заявления, направляет
        уведомление руководству учебного заведения;
        \item руководство учебного заведения фиксирует отмену заявления;
        \item система обновляет список абитуриентов, подавших заявление в ВУЗ.
      \end{enumerate}
  \end{enumerate}

  \textbf{Специальные требования}:
  \begin{itemize}
    \item все документы и информация о абитуриентах должны быть защищены и
    храниться в зашифрованном виде;
    \item система должна выдерживать высокие нагрузки: использование системы
    большим количеством пользователей единовременно не должно препятствовать
    работе системы;
    \item при сбое системы никакие данные не должны быть повреждены, все
    операции, выполняемые во время сбоя, должны быть отменены без повреждения
    данных.
  \end{itemize}

  \textbf{Список технологий и типов данных}:
  \begin{itemize}
    \item доступ систему должен быть возможен со всех основных типов устройств и
    операционных систем (iOS, Android, Windows, Linux, macOS);
    \item база данных, используемая для хранения данных об абитуриентах, должна
    соответствовать требованиям ACID, а также обладать механизмом транзакций.
  \end{itemize}

  \textbf{Частота использования}:
  \begin{itemize}
    \item для абитуриентов: от 1 до 5 раз за все время;
    \item для руководств учебных заведений: ежегодно во время работы приемной
    кампании.
  \end{itemize}

  \textbf{Открытые вопросы}:
  \begin{itemize}
    \item изучить нормативные документы по организации процессов приемных
    кампаний разных ВУЗов;
    \item исследовать максимально возможную нагрузку в пик подачи заявлений во
    время приемной кампании;
    \item выяснить, какое количество данных будет хранить система.
  \end{itemize}
}

\end{document}
